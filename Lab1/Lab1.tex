\documentclass[12pt, letterpaper, titlepage]{article}
\usepackage[utf8]{inputenc}
\usepackage{geometry}
\usepackage{color,graphicx,overpic} 
\usepackage{fancyhdr}
\usepackage{amsmath,amsthm,amsfonts,amssymb}
\usepackage{mathtools}
\usepackage{hyperref}
\usepackage{multicol}
\usepackage{array}
\usepackage{float}
\usepackage{blindtext}
\usepackage{longtable}
\usepackage{scrextend}
\usepackage[font=small,labelfont=bf]{caption}
\usepackage[framemethod=tikz]{mdframed}
\usepackage{calc}
\usepackage{titlesec}
\usepackage{listings}
\usepackage[normalem]{ulem}
\usepackage{tabularx}
\usepackage{mathrsfs}
\usepackage{bookmark}
\usepackage{setspace}
\usepackage{tabularx}
\usepackage{ltablex}
\usepackage{enumerate}

\mathtoolsset{showonlyrefs}  
\allowdisplaybreaks

\definecolor{mycolor}{rgb}{0, 0, 0}

\geometry{top=2.54cm, left=2.54cm, right=2.54cm, bottom=2.54cm} 
\setlength{\headheight}{20pt}
\setlength{\parskip}{0.3cm}
\setlength{\parindent}{1cm}

\pagestyle{fancy}
\fancyhf{}
\rhead{Lora Ma - 1570935}
\lhead{\textit{ECE 493 Lab 1}}
\rfoot{Page \thepage}

\begin{document} 
\singlespacing
\section*{Q1}
\begin{enumerate}
    \item I think a lot of the analysis in the paper is still true except for a few points. For example, the paper states that "Groups have been known to be cruel and unjust to their deviant members." which is untrue today. Most organizations today have come up with civil and standardized methods to deal with this and it isn't really a problem today. In fact, large organizations sometimes seek to diversify their members to have more different perspectives. Another example is that the paper states that if an organization consists of groups, there will be a lot of infighting between groups and that there will be more conflict issues. I believe that this is untrue in my experience. Groups that are sufficiently large have many very specialized groups and those groups rarely interact with other groups. In industry, we can see that groups actually can work very well together.
    \item I have never been part of a highly individualistic organization in my life, but I have been part of groups with many highly individualistic people. In these situations, there was often a lot more fighting as people were stubborn in their own ideas and methodologies. In my opinion, a "groupy" organization might have been more effective at accomplishing goals because it would foster more collaboration and teamwork. In a highly individualistic organization, it can be difficult for people to work together towards a common goal because they may prioritize their own agendas over the goals of the organization. However, it's worth noting that every organization is different and what works well for one organization may not be as effective in another.
    \item I have been part of several groupy organizations in my life. While the motivation of a highly individualistic individual could potentially result in getting a lot of work done quickly, I find that a groupy organization is generally much more pleasant to work in. In these organizations, although it may take a bit longer and require more collaboration, eventually the organization's goals are accomplished. I think that a highly individualistic organization might not be as effective at accomplishing goals because it could lead to more conflict and a lack of teamwork. In a groupy organization, people are more likely to work together and support each other towards the common goal, which can be more effective in the long run. However, again, it's important to consider that every organization is different and what works well for one organization may not be as effective in another.
    \item It's important to note that generalizations about gender and their effects on team dynamics should be made with caution, as individuals can vary widely in their attitudes and behaviors regardless of their gender. That being said, research has shown that there are some differences in the way that men and women tend to communicate and interact in groups. For example, some studies have found that women tend to be more collaborative and relationship-focused in their communication, while men tend to be more competitive and individualistic. In my own social circle, as a female, I am one of the most competitive out of my peers while I have male friends that are more socialized to have more relational attitudes. 
\end{enumerate}

\section*{Q2}
\begin{enumerate}
    \item \textbf{Context} refers to the environment or situation in which something may be considered. It often includes outside factors such as the goal of the task, the location, or the people involved in the task. Understanding context is very important in fully understanding a situation. \textbf{Workflow} refers to the sequence of steps, processes, or procedures required to complete a project. \textbf{Levels} refers to different stages or complexity levels in a system pr process. In general, it refers to the hierarchical structure of something. \textbf{Time} refers to the progression of events usually measured in units such as seconds, minutes, hours, or days.
    \item The instructor's decision to keep teams to a size of three members may reflect their belief that smaller groups are more efficient and effective than larger groups. In smaller groups, members tend to have more opportunity to contribute and be heard, which can lead to more productive and collaborative work. Additionally, it may be easier to coordinate schedules and ensure that all group members are completing their work on time in a group of three.
    \item The five factors are openness, conscientiousness, extraversion, agreeableness, and neuroticism. \textbf{openness} reflects an individual's curiosity, imagination and willingness to try new things. Usually people who have a high score in openness are creative, curious, and open-minded while people who score low are more stubborn. \textbf{Conscientiousness} reflects an individual's organization level, reliability level, and self-discipline. People who have a high score are responsible and dependable while people who score low are disorganized. \textbf{Extraversion} reflects an individual's sociability level, assertiveness level, and self-discipline level. Usually people who score high are outgoing and sociable while people who score low are more introverted, quiet, and reserved. \textbf{Agreeableness} reflects an individual's likability, cooperation, and compassion level. People who have a high score are friendly and cooperative while people who have a low score are more competitive. \textbf{Neuroticism} reflects an individual's level of emotional instability such as their levels of anxiety and moodiness. People who have a high score are anxious and oftentimes insecure while people who score low are more emotionally resilient. 
    \item Tuckman's four stages of a team's life-cycle is forming, storming, norming, and performing. During forming, the team is formed and the group task is discussed. Hierarchical structure structure is also established between members and a method for how to finish the group task is discussed. Oftentimes it  can be difficult to define the group task, so this can lead to the storming stage. During the storming stage, team members debate over what actions should be taken and conflict arises. Members will struggle to reconcile competing loyalties and responsibilities. During the norming stage, members will reduce emotional conflict and start to become more cooperative. This will develop a bit of cohesion within the group with some ground rules, outlined roles, and status. During the performing stage, members are able to work through conflict. They become attached to the team and satisfied with the progress of the team as they complete their task.
\end{enumerate}
\section*{Q3}
\begin{enumerate}
    \item In the text provided, the "innovation dividend" is described as the benefit that scientific organizations may obtain from encouraging more gender diversity among their teams. This is because diverse teams, including those with a balance of genders, can lead to "smarter, more creative teams" which can open the door to new discoveries. The text suggests that under the right conditions, teams could benefit from different and many types of diversity and researchers have already found positive links between gender diversity and collective productivity.
    \item There are several explanations outlined in the text for why gender diversity can lead to an "innovation dividend":
    \begin{itemize}
        \item Studies show that a collective intelligence factor predicts group performance better than the individual IQ members of the group members. Women typically have a higher level of social perceptiveness and a key component of this collective intelligence factor is the group member's social perceptiveness. 
        \item Gender diversity can lead to new discoveries by broadening the group's viewpoints. Studies show that scholarly contributions written by women-dominated author groups typically pose different questions and engage in different topics than men-dominated author groups. This suggests that the same difference in perspective can apply to research. 
    \end{itemize}
    \item "Critical mass" is referred to as a minimum level of representation of a particular group of people within a team that is necessary to have meaningful impact. It is usually used for underrepresented groups and can be used to counteract effects of discrimination or bias. By reaching critical mass for underrepresented groups, it can lead to better, more productive, and more creative teams with a wider range of perspectives.
    \item It is suggested that achieving a critical mass in representation of women (or other underrepresented groups) in teams is necessary for the team to experience less stereotyping, more involvement in decision making and teamwork, and higher levels of support. The text suggests that this critical mass level is necessary to overcome the barriers that women face in male-dominated fields and to change the culture and dynamics within the team and the organization, but it doesn't provide any explanation for why the specific range of 15\% to 30\% is necessary. 
\end{enumerate}
\section*{Q4}
\begin{enumerate}
    \item DISC is a personality model that is used to access a person's personality and behaviour. DISC stands for the four personality traits Dominance, Influence, Steadiness, and Compliance. Each person will have a unique combination of the 4 traits which will make up their personality and behaviour. \textbf{Dominance} refers to a person's assertiveness, competitiveness, and desire to take charge. They are usually confident and assertive, but aggressive and impatient. \textbf{Influence} refers to a person's sociability and communication skills. They are usually enthusiastic but impulsive. \textbf{Steadiness} refers to a person's patience, consistency, and reliability. They could also be seen as reserved and inflexible. \textbf{Compliance} refers to a person's attention to detail and compliance with rules and procedures. They are usually rule followers that are detail-oriented but rigid.
    \item The study found that project managers with D-type personalities had higher success scores and that teams with a higher percentage of I-type members also had higher success scores. However, having too many D-type members on a team was found to lower the success score. The study also found that increasing C-type members decreased schedule scores, while increasing members with leader-type personalities led to higher schedule scores. There was no correlation found between personality types and effort or external risk. Additionally, an increase in I-type members likely increased internal issue scores, while D-type and C-type decreased internal issue scores. D-type members decreased software quality scores, while I-type members increased software quality and customer satisfaction scores.
    \item 
    \begin{itemize}
        \item Hofstede's Cultural Dimensions is used to compare cultural differences between countries. It has six dimensions: power distance, individualism verses collectivism, masculinity verses femininity, uncertainty avoidance, long-term orientation, and indulgence versus restraint. These dimensions can be used as a guide for groups working in cross-cultural environments. Power distance looks at the extent to which less powerful members of society accept unequal distribution of power. Individualism versus collectivism looks at the extent individuals in a society are integrated into groups and how they value being part of groups and collective goals. Masculinity versus femininity is the level of importance that is placed on traditional masculine or traditional female masculine values. Uncertainty avoidance is the level at which members of a society feels threatened by uncertainty. Long-term orientation is the level to which members of society view long-term goals and perseverance. Indulgence verses restraint is the level at which members of society enjoy life. 
        \item \begin{itemize}
            \item Power distance: China has a high score in this area meaning that less powerful members of society greatly accept unequal distribution of power. This results in a large gap between classes in society and oftentimes strong traditions and respect for authority.
            \item Individualism versus collectivism: China scores low in individualism which means that people will lean towards being in groups and collective goals rather than individual goals.
            \item Masculinity versus femininity: China scores high in masculinity which means that traditional masculine and feminine values and roles are valued in society 
            \item Uncertainty avoidance: China scores in the middle for uncertainty avoidance which means that the members of society are comfortable with some level of uncertainty
            \item Long-term orientation: China scores high in this area which means society values long-term goals and perseverance.
            \item Indulgence versus restraint: China scores highly in restraint which means that they control their desires and enjoy life.
        \end{itemize}
        \item Hofstede's cultural dimensions and the DISC personality model are not directly related since they measure different aspects of culture and personality; however, Hofstede's dimensions may provide a general idea about the cultural background of an individual depending on their background and whether the society they were in may have influenced his/her personality, values, and behaviour. Despite this, there are still some relations. For example, 
    \end{itemize}
\end{enumerate}
\end{document}  